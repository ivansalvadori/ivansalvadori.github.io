% Ao menos uma linguagem (brazil ou english) deveria sempre ser fornecida
\documentclass[brazil,aspectratio=169]{lapesd-slides}

\usepackage{pgfgantt}

%%%%%%%%%%%%%%%%%%%%%%%%%%%%
% Metadados
%%%%%%%%%%%%%%%%%%%%%%%%%%%%

\title[]{Infraestruturas de software para geração e publicação de dados conectados na Web}
\subtitle{}
\author[Ivan Salvadori]{\large Ivan Salvadori \\{\small \texttt{ivanlsalvadori@gmail.com}}}

\institute{
  \fontsize{10.5}{12.6}\selectfont 
  Programa de Pós-Graduação em Ciência da Computação\\ 
  Depto. de Informática e Estatísitca\\
  Universidade Federal de Santa Catarina - Florianópolis\\
  \vspace{1em}
  
}

\date{\today}

%%%%%%%%%%%%%%%%%%%%%%%%%%%%
% Slides
%%%%%%%%%%%%%%%%%%%%%%%%%%%%

\begin{document}

\titleframe

% Você não é obrigado a colocar um sumário!
\begin{frame}{Sumário}
  \tableofcontents
\end{frame}

% Desse ponto em diante serão inseridos slides de pausa a cada \section
\showsections
\section{Introdução}
\subsection{Web Semântica}

\begin{frame}{Web Semântica}
	\begin{block}{A visão de Tim Berners-Lee}
	    Um lugar onde os processos automatizados podem ser realizados. Isso significa publicar documentos e dados na web de forma que possam ser interpretados, integrados, agregados e consultados para revelar novas conexões e responder perguntas, ao invés de apenas navegar e pesquisar.
	\end{block}

\end{frame}


\begin{frame}{Web Semântica}
	\begin{block}{A visão de Tim Berners-Lee}
		\begin{itemize}
			\item Padronização
			\item Metadados
			\item Automação
			\item Reuso de informação
			\item Inferência
			\item Interoperabilidade
			\item Serendipidade
		\end{itemize}		
	\end{block}
	
\end{frame}

\begin{frame}{Web Semântica}
	\begin{block}{A visão de Tim Berners-Lee}
		André Santanchè (UNICAMP) -> Ciência da Web
		\begin{itemize}
			\item Web como plataforma -> Web começa a ficar pervasiva
		\end{itemize}
		\begin{enumerate}
			\item Multimidia
			\item Storage 
			\item Uso offline de aplicações 
			\item Acesso a dispositivos
			\item Integração 
		\end{enumerate}
	\end{block}
	
\end{frame}





\subsection{Dados Conectados}
\section{Plataformas para Publicação de Dados Conectados}
\section{Alternativas}
\subsection{Foco em conectar dados}
\subsection{Foco em reutilizar dados}
\subsection{Foco em interoperabilidade}
\section{Capacidade Semântica?}
\section{Conclusão}













%%%%%%%%%%%%%%%%%%%%%%%%%%%%
% Finalização
%%%%%%%%%%%%%%%%%%%%%%%%%%%%

\thanksframe
\referencesframe{userguide}

\begin{backup}
  \pholder{Slide de backup}
\end{backup}

\end{document}

% LocalWords:  template cls standalone GitHub Overleaf bugfixes SVGs
% LocalWords:  Re-empacotamento fontsize Makefile pdflatex imgs PDFs
% LocalWords:  shell-escape frames SVG brazil english lapesd-slides
% LocalWords:  disabletodonotes todonotes TODO's backup showbackup
% LocalWords:  hidebackup abntexcite abntex natbib nobib titleframe
% LocalWords:  frame showsections sidebar stopcountingframes default
% LocalWords:  thanksframe Thank You Questions referencesframe titulo
% LocalWords:  bibfiles pholder todonote placeholder inline addfig
% LocalWords:  opts graphicx addfiglw width Citations dijkstra Direct
% LocalWords:  Closure Parallel dynamic scheduling DoImportantStuff
% LocalWords:  lccp merged cell svg pdf
